% \textbf{Nomenclature}

% \begin{itemize}
% \item $\mathbb{R}, \mathbb{N}$: the sets of real and natural numbers.
% \item calligraphic uppercase letters ${\cal S}, {\cal U}, {\cal P}$: sets,
% \item sans serif uppercase letters $\mathsf{U}$, $\mathsf{E}$:
%   statements,
% \item uppercase letters $A, B, P$: points,
% \item lowercase boldfaced letters $\mathbf{a}, \mathbf{v}$: vectors,
% \item lowercase letters $x,y,c,d$: scalars, or set elements,
% \item lowercase letters $m, n$: natural numbers
% \end{itemize}


%We start from the first three items of Section \ref{sec:sets}.

% \begin{rem}
%   Math is the science of patterns.
%   These patterns are useful in science and engineering.
%   In addition, sometimes these patterns are also useful
%   for the very process of math learning.
%   This chapter concerns the footprint of subsequent chapters
%   in sets and logic, in basic analysis,
%   in linear algebra and in abstract algebra.
%   We also emphasize some key ideas behind
%   a number of common notions.
% \end{rem}

\begin{rem}
The main ingredients of snacks are sugar and fat.
The main ingredients of math are logic and set theory.
One common saying in computer science is 
  ``If syntax sugar does not count, there is nothing left.''
  This is also true for mathematics. 
  All high-level mathematical concepts
  can be considered as syntax sugars
  of logic and set theory.
\end{rem}

\section{Set-theoretic rules}
\label{sec:sets-set-theoretic}

\begin{defn}
  \label{def:setNotation}
  A \emph{set} ${\cal S}$
  is a collection of \emph{distinct} objects
   that share a common quality; 
   it is often denoted with the following notation
   \begin{equation}
     \label{eq:setNotation}
     {\cal S} = \{ x\ :\ \text{ the conditions that $x$ satisfies } \}.
   \end{equation}
   $x\in {\cal S}$ means that the object $x$ belongs to ${\cal S}$;
   otherwise we write $x\not\in {\cal S}$. 
\end{defn}

\begin{ntn}
$\mathbb{R}, \mathbb{Z}, \mathbb{N}, \mathbb{Q}, \mathbb{C}$
 denote 
 the sets of real numbers, integers, natural numbers,
 rational numbers and complex numbers, respectively.
$\mathbb{R}^+, \mathbb{Z}^+, \mathbb{N}^+, \mathbb{Q}^+$
the sets of positive such numbers.
In particular, $\mathbb{N}$ contains the number zero while
 $\mathbb{N}^+$ does not.
\end{ntn}

\begin{defn}
  \label{def:subsets}
  ${\cal S}$ is a \emph{subset} of ${\cal U}$,
  written ${\cal S}\subseteq {\cal U}$,
  if and only if (iff) $x\in {\cal S}$ $\Rightarrow$ $x\in {\cal U}$.
  ${\cal S}$ is a \emph{proper subset} of ${\cal U}$,
  written ${\cal S}\subset {\cal U}$,
  if ${\cal S}\subseteq {\cal U}$
  and $\exists x\in {\cal U}$ s.t. $x\not\in{\cal S}$.
\end{defn}

\begin{defn}
  \label{eq:AminusB}
  The \emph{difference of two sets} is 
  \begin{equation}
    \label{eq:AminusB}
    A\setminus B := \{ x : x\in A \text{ and } x\not\in B \}; 
  \end{equation}
  it is also called the \emph{complement of $B$ in $A$}. 
\end{defn}

\begin{defn}
  \label{eq:AcapB}
  The \emph{intersection of two sets} is 
  \begin{equation}
    \label{eq:AcapB}
    A\cap B := \{ x : x\in A \text{ and } x\in B \}.
  \end{equation}
\end{defn}

\begin{defn}
  \label{eq:AcupB}
  The \emph{union of two sets} is 
  \begin{equation}
    \label{eq:AcupB}
    A\cup B := \{ x : x\in A \text{ or } x\in B \}.
  \end{equation}
\end{defn}

\begin{thm}[Distributive laws]
  \label{thm:distributiveLaws}
  For any sets $A$, $B$, and $C$, we have
  \begin{align}
    \label{eq:distributiveLaw1}
    A\cap (B\cup C) &= (A\cap B) \cup (A\cap C),
    \\
    \label{eq:distributiveLaw2}
    A\cup (B\cap C) &= (A\cup B) \cap (A\cup C).
  \end{align}
\end{thm}

\begin{thm}[DeMorgan's laws]
  \label{thm:DeMorganLaws}
  For any sets $A$, $B$, and $C$, we have
  \begin{align}
    \label{eq:DeMorganLaw1}
    A\setminus (B\cup C) &= (A\setminus B) \cap (A\setminus C),
    \\
    \label{eq:DeMorganLaw2}
    A\setminus (B\cap C) &= (A\setminus B) \cup (A\setminus C).
  \end{align}
\end{thm}


\section{First-order logic}
\label{sec:logic}

\begin{defn}[Statements of first-order logic]
\label{def:uni_exist}
A \emph{universal statement} is a logical statement 
 of the form
\begin{equation}
  \mathsf{U} = (\forall x\in {\cal S},\ \mathsf{A}(x) ).
\end{equation}
An \emph{existential statement} has the form
\begin{equation}
  \mathsf{E} = (\exists x\in {\cal S},\text{ s.t. } \mathsf{A}(x)),
\end{equation}
 where 
 $\forall$ (``for each'') and $\exists$ (``there exists'')
 are the \emph{quantifiers}, ${\cal S}$ is a set,
 ``s.t.'' means ``such that,''
 and $\mathsf{A}(x)$ is the \emph{formula}.\\
A statement of \emph{implication/conditional}
 has the form
 \begin{equation}
   \mathsf{A}\Rightarrow \mathsf{B}.
 \end{equation}
\end{defn}

 \begin{exm}
   Universal and existential statements:\\
   $\forall x\in[2,+\infty)$, $x>1$;\\
   $\forall x\in \mathbb{R}^+$, $x>1$;\\
   $\exists p,q\in \mathbb{Z}, \text{ s.t. } p/q = \sqrt{2}$;\\
   $\exists p,q\in \mathbb{Z}, \text{ s.t. } \sqrt{p} = \sqrt{q}+1$.
 \end{exm}

\begin{defn}
  \emph{Uniqueness quantification}
   or \emph{unique existential quantification},
   written $\exists!$ or $\exists_{=1}$, 
   indicates that exactly one object with a certain property exists.
\end{defn}

\begin{exc}
  Express the logical statement $\exists! x, \text{ s.t. } \mathsf{A}(x)$
   with $\exists$, $\forall$, and $\Leftrightarrow$.
\end{exc}
\begin{solution}
  $\exists x \text{ s.t. }\forall y, \mathsf{A}(y) \Leftrightarrow x=y.$
\end{solution}

 \begin{rem}
A logical statement is either true or false.
There is no such thing that
 a logical statement is sometimes true and sometimes false.
To prove a universal statement,
 conceptually we have to verify the statement
 for all elements in the set.
To deny a universal statement,
 we only need to show a counterexample.
To prove an existential statement,
 we only need to show an instance.
To deny an existential statement,
 conceptually we have to show that the statement holds
 for none of the elements.
 \end{rem}

 \begin{rem}
   \label{rem:firstOrderLogicAndPropLogic}
   Propositional logic is the study
   of the propositional connectives
   \emph{not}, \emph{or}, \emph{and},
   \emph{if $\cdots$ then},
   and \emph{if and only if},
   respectively denoted by $\neg$,
   $\vee$, $\wedge$, $\Rightarrow$,
   and $\Leftrightarrow$.
   Propositional logic is not sufficient
   for all math courses,
   so we need a more advanced branch
   of logic, called
   \emph{first-order logic},
   which includes proposition logic,
   but in addition uses the quantifiers
   $\forall$ and $\exists$ in Definition \ref{def:uni_exist}.
   See the book by \cite{hodel13:_introd_mathem_logic}
   for more on mathematical logic.
 \end{rem}
 
 \begin{rem}
   In Definition \ref{def:uni_exist},
    the formula $\mathsf{A}(x)$ itself
    might also be a logical statement.
   Hence universal and existential statements
    might be nested.
   This observation leads to the next definition.
 \end{rem}

 \begin{defn}
   A \emph{universal-existential statement} is a logical statement 
   of the form
   \begin{equation}
     \mathsf{U}_E =
     (\forall x\in {\cal S},\ \exists y\in {\cal T}
     \text{ s.t. } \mathsf{A}(x,y)).
   \end{equation}
   An \emph{existential-universal statement} has the form
   \begin{equation}
     \mathsf{E}_U =
     (\exists y\in {\cal T},\text{ s.t. } \forall x\in {\cal S},\ 
     \mathsf{A}(x,y)).
   \end{equation}
 \end{defn}

 \begin{exm}
   True or false:\\
   $\forall x\in[2,+\infty)$, $\exists y\in \mathbb{Z}^+$ s.t. $x^y<10^5$;\\
   $\exists y\in \mathbb{R}$ s.t.
   $\forall x\in[2,+\infty)$, $x>y$;\\
   $\exists y\in \mathbb{R}$ s.t.
   $\forall x\in[2,+\infty)$, $x<y$.
 \end{exm}

\begin{exm}
  [Translating an English statement into a logical statement]
  Goldbach's conjecture states
   \emph{every even natural number greater than 2
     is the sum of two primes}.
  Let $\mathbb{P}\subset \mathbb{N}^+$
   denote the set of prime numbers.
  Then Goldbach's conjecture is
  $\forall a\in 2\mathbb{N}^++2,
   \exists p,q\in \mathbb{P}$, s.t. $a=p+q$.
 \end{exm}
 
 \begin{thm}
   \label{thm:existUnivImpliesUnivExist}
   The existential-universal statement
    implies the corresponding universal-existential statement,
    but not vice versa.
 \end{thm}

 \begin{exm}[Translating a logical statement to an English statement]
   Let ${\cal S}$ be the set of all human beings.\\
   $U_E=$($\forall p\in{\cal S}, \exists q\in{\cal S}$ s.t. $q$ is $p$'s mom.)
   \\
   $E_U$=( $\exists q\in{\cal S}$ s.t.
   $\forall p\in{\cal S}, $ $q$ is $p$'s mom.)\\
   $U_E$ is probably true, but $E_U$ is certainly false. \\
   If $E_U$ were true, then $U_E$ would be true. Why?
 \end{exm}

\begin{axm}[First-order negation of logical statements]
  The negations of the statements in Definition \ref{def:uni_exist}
  are
  \begin{align}
  \neg \mathsf{U} &= (\exists x\in {\cal S},\text{ s.t. }
  \neg \mathsf{A}(x)).
  \\
  \neg \mathsf{E} &= (\forall x\in {\cal S},\ 
  \neg \mathsf{A}(x)).
  \end{align}
\end{axm}

\begin{rul}
  The negation of a more complicated logical statement
   abides by the following rules:
\begin{itemize}\itemsep0em
\item switch the type of each quantifier until
  you reach the last formula without quantifiers;
\item negate the last formula.
\end{itemize}
In particular, the negation of an implication formula
$P \Rightarrow Q$,
 is $P \wedge \neg Q$.
\end{rul}

\begin{rem}
  An $n$-ary \emph{truth function} is a function
  $H: \{T, F\}^n \rightarrow \{T,F\}$
  where $T$ and $F$ stand for ``true'' and ``false,'' respectively.
  To each connective in Remark \ref{rem:firstOrderLogicAndPropLogic},
  we assign a truth function as its characterization.
  The truth functions for ``$\neg, \vee, \wedge, \Leftrightarrow$''
  are straightforward while that for ``$\Rightarrow$''
  needs some explanation.
  \begin{center}
    \begin{tabular}{cc|c}
      $A$ & $B$ & $A \Rightarrow B$
      \\ \hline
      T & T & T
      \\ \hline
      T & F & F
      \\ \hline
      F & T & T
      \\ \hline
      F & F & T
      \\ \hline
    \end{tabular}
  \end{center}
  In the above truth table, 
  values of $H_{\Rightarrow}$
  for the first two cases follow directly from
  the meaning of the connective \emph{if $\ldots$ then}; 
  it is only these two cases that involve
  the proof of $A \Rightarrow B$
  under the condition of $A$ being true.
  But what if $A$ is false?
  We have four possibilities
  for the last two cases
  and the choices of the values of $H_{\Rightarrow}$
  in the above truth table
  are inevitable because any other choice
  would coincide with that of $\wedge$, $\Rightarrow$,
  or $B$.

  It follows from the above discussions that
  the negation of $P \Rightarrow Q$
  is $P \wedge \neg Q$.
\end{rem}

\begin{rem}
  We prefer to write logic statements in symbols
  instead of words, 
  as this facilitates mathematical reasoning
  and helps memorization.
  Also, one might need to group quantifiers of the same type.
\end{rem}

\begin{exm}
  [The negation of Goldbach's conjecture]
  $\exists a\in 2\mathbb{N}^++2$ s.t. $\forall p,q \in \mathbb{P}$, 
   $a\ne p+q$.
\end{exm}

\begin{rem}
 Goldbach's conjecture has been shown to hold up through $4\times10^{18}$,
   but no proofs and disproofs have been found.
\end{rem}

\begin{exc}
 Negate the logical statement in Definition \ref{def:RiemannIntegrable}.
\end{exc}
   % \begin{exc}
   %   A weaker version of Goldbach's conjecture states
   %   \emph{Goldbach's conjecture has
   %     at most a finite number of of counterexamples}.
   %   Formulate it into a logical statement
   %    with explicit quatifiers.
   % \end{exc}

   % \begin{exc}
   %   The only even prime is 2.\\
   %   Multiplication of integers is associative. \\
   %   Every positive integer has a unique prime factorization.
   % \end{exc}

\begin{axm}[Contraposition]
  \label{axm:contrapositive}
  A conditional statement is logically equivalent to its
  contrapositive.
  \begin{equation}
    \label{eq:contraposition}
    (\mathsf{A}\Rightarrow \mathsf{B}) \Leftrightarrow
    (\neg \mathsf{B}\Rightarrow \neg \mathsf{A})
  \end{equation}
\end{axm}

\begin{exm}
  \label{exm:contrapositive}
  ``If Jack is a man, then Jack is a human being.''
  is equivalent to ``If Jack is not a human being,
  then Jack is not a man.''
\end{exm}

\begin{exc}
  Draw an Euler diagram of subsets to illustrate Example \ref{exm:contrapositive}.
\end{exc}

\begin{exc}
  Rewrite each of the following
  statements and its \emph{negation}
  into \emph{logical statements}
  using symbols, quantifiers, and formulas.
  \begin{enumerate}[(a)]\itemsep0em
  \item The only even prime is 2.
  \item Multiplication of integers is associative.
  \item Goldbach's conjecture has
    at most a finite number of counterexamples.
  \end{enumerate}  
\end{exc}
\begin{solution}
\begin{enumerate}[(a)]
\item Let $\mathbb{P}$ denote the set of all prime numbers.

\[
\mathcal{G}=(\mathbb{P}\cap 2\mathbb{Z}=\{2\});
\]
\[
\neg\mathcal{G}=(\mathbb{P}\cap 2\mathbb{Z}\ne\{2\}).
\]

\item\[
\mathcal{G}=(\forall x,y,z \in \mathbb{Z},\  x(yz)=(xy)z);
\]
\[
\neg\mathcal{G}=(\exists x,y,z \in \mathbb{Z}\text{ s.t. } x(yz)\neq(xy)z).
\]
\item Let $\mathbb{E}= 2\mathbb{N^+}\setminus \{2\}$.

Let $\mathbb{P}$ be the set of all prime numbers.
\begin{align*}
\mathcal{G} =\bigl(&\exists n\in\mathbb{N}^+\text{ s.t. }
 \forall x\in\mathbb{E},\\  
 &(x>n)\ \Rightarrow\ (\exists p,k\in\mathbb{P}\text{ s.t. } x= p+k)
 \bigr);  \\
\neg\mathcal{G} =\bigl(&\forall n\in\mathbb{N}^+,
 \exists x\in\mathbb{E}\text{ s.t. } \\
 &(x>n)\ \Rightarrow\ (\forall p,k\in\mathbb{P},\  x\ne p+k)
 \bigr).
\end{align*}
\end{enumerate}
\end{solution}


\section{Ordered sets}
\label{sec:sets}

\begin{defn}
  \label{def:CartesianProduct}
  The \emph{Cartesian product} ${\cal X}\times {\cal Y}$
   between two sets ${\cal X}$ and ${\cal Y}$
   is the set of all possible ordered pairs with first element
   from ${\cal X}$ and second element from ${\cal Y}$:
   \begin{equation}
     {\cal X}\times {\cal Y} = \{(x,y)\ |\ x\in {\cal X},\ y\in {\cal Y}\}.
   \end{equation}
\end{defn}

\begin{axm}[Fundamental principle of counting]
  \label{axm:multiplicationPrinciple}
  Consider a task that consists of a sequence of $k$ independent steps.
  Let $n_i$ denote the number of different choices for the $i$-th step,
   the total number of distinct ways to complete the task
   is 
   \begin{equation}
     \prod_{i=1}^{k} n_i = n_1n_2\cdots n_k.
   \end{equation}
 \end{axm}

 \begin{exm}
   \label{exm:dinnerCombos}
   Let $A, E, D$ be the set of appetizers,
    main entrees, desserts in a restaurant.
   $A\times E\times D$
    is the set of possible dinner combos.
   If $\#A=10$, $\#E=5$, $\#D=6$,
    $\#(A\times E\times D)=300$.
 \end{exm}

 \begin{rem}
   After being seated at a restaurant table,
   you will feel weird if the waiter
   show you a menu consisting of all the combos
   explicitly spelled out.
   % Some people would even regard it as an insult of their
   % intelligence.
   This illustrates that we should employ Cartesian product
    in a functional way to increase the utility
    of the math knowledge we learned in a hard way.
 \end{rem}
 
\begin{defn}[Maximum and minimum]
  Consider ${\cal S}\subseteq \mathbb{R}$,
   ${\cal S}\ne \emptyset$.
  If $\exists s_m\in{\cal S}$
   s.t. $\forall x\in {\cal S}$, $x\le s_m$,
   then $s_m$ is the \emph{maximum} of ${\cal S}$
   and denoted by $\max {\cal S}$.
  If $\exists s_m\in{\cal S}$
   s.t. $\forall x\in {\cal S}$, $x\ge s_m$,
   then $s_m$ is the \emph{minimum} of ${\cal S}$
   and denoted by $\min {\cal S}$.
\end{defn}

\begin{defn}[Upper and lower bounds]
  Consider ${\cal S}\subseteq \mathbb{R}$,
   ${\cal S}\ne \emptyset$.
  $a$ is an \emph{upper bound} of ${\cal S}\subseteq \mathbb{R}$
  if $\forall x\in {\cal S}$, $x\le a$;
   then the set ${\cal S}$ is said to be \emph{bounded above}.
  $a$ is a \emph{lower bound} of ${\cal S}$
   if $\forall x\in {\cal S}$, $x\ge a$;
   then the set ${\cal S}$ is said to be \emph{bounded below}.
  ${\cal S}$ is \emph{bounded}
   if it is bounded above and bounded below.
\end{defn}

\begin{rem}
One difference between a maximum and an upper bound
 is that the former belongs to the set
 while the latter might not.
Another difference is that,
 for a bounded interval,
 the upper bound always exists
 while the maximum might not exist.
\end{rem}

\begin{defn}[Supremum and infimum]
  \label{def:SupAndInf}
  Consider a nonempty set \mbox{${\cal S}\subseteq \mathbb{R}$}.
  If ${\cal S}$ is bounded above and ${\cal S}$
   has a least upper bound 
   then we call it the \emph{supremum}
   of ${\cal S}$
   and denote it by $\sup {\cal S}$.
  If ${\cal S}$ is bounded below and ${\cal S}$
   has a greatest lower bound,
   then we call it the \emph{infimum}
   of ${\cal S}$
   and denote it by $\inf {\cal S}$.
\end{defn}

\begin{exm}
  If a set ${\cal S}\subset \mathbb{R}$ has a maximum,
   we have \mbox{$\max{\cal S}=\sup {\cal S}$}.
\end{exm}

\begin{exm}
  $\sup[a,b]=\sup[a,b)=\sup(a,b]=\sup(a,b)$.
\end{exm}

\begin{thm}[Existence and uniqueness of least upper bound]
  \label{thm:existenceAndUniquenessOfLeastUpperBound}
  Every nonempty subset of $\mathbb{R}$ %${\cal S}\subseteq \mathbb{R}$
   that is bounded above has exactly one least upper bound.
\end{thm}

\begin{rem}
  Theorem \ref{thm:existenceAndUniquenessOfLeastUpperBound} states that, 
  for any nonempty ${\cal S}\subseteq \mathbb{R}$ bounded above,
  $\sup {\cal S}$ exists and is a real number.
\end{rem}

\begin{coro}
  \label{coro:existenceAndUniquenessOfGreatestLowerBound}
  Every nonempty subset of $\mathbb{R}$ %${\cal S}\subseteq \mathbb{R}$
  that is bounded below has a greatest lower bound.
\end{coro}

\begin{defn}
  A \emph{binary relation between two sets} ${\cal X}$ and ${\cal Y}$
  is an ordered triple
  (${\cal X}, {\cal Y}, {\cal G}$)
  where ${\cal G}\subseteq{\cal X}\times{\cal Y}$.\\
  % or equivalently, a map
  % $R: {\cal X}\times{\cal Y}\rightarrow {\cal G}$.\\
  A \emph{binary relation on} ${\cal X}$
  is the relation between ${\cal X}$ and ${\cal X}$.\\
  The statement $(x,y)\in R$ is read
  ``$x$ is $R$-related to $y$,'' and
  denoted by $xRy$ or $R(x,y)$.
\end{defn}

\begin{defn}
  An \emph{equivalence relation} ``$\sim$'' on ${\cal A}$ is 
  a binary relation on ${\cal A}$ 
  that satisfies
  $\forall a,b,c\in{\cal A}$,
  \begin{itemize}
    \itemsep0em
  \item $a\sim a$ (reflexivity);
  \item $a\sim b$ implies $b\sim a$ (symmetry);
  \item $a\sim b$ and $b\sim c$ imply $a\sim c$ (transitivity).
  \end{itemize}
\end{defn}

\begin{defn}
  \label{defn:totalOrder}
  A binary relation ``$\le$'' on some set ${\cal S}$
  is a \emph{total order} or \emph{linear order} on ${\cal S}$
  iff,
  $\forall a,b,c\in{\cal S}$,
  \begin{itemize}
  \item $a\le b$ and $b\le a$ imply $a=b$ (antisymmetry);
  \item $a\le b$ and $b\le c$ imply $a\le c$ (transitivity);
  \item $a\le b$ or $b\le a$ (totality).
  \end{itemize}
  A set equipped with a total order
  is a \emph{chain} or \emph{totally ordered set}.
\end{defn}

\begin{exm}
  The real numbers with less or equal.
\end{exm}

\begin{exm}
  The English letters of the alphabet with dictionary order.
\end{exm}

\begin{exm}
  The Cartesian product of a set of totally ordered sets
  with the \emph{lexicographical order}.
\end{exm}

\begin{exm}
  Sort your book in lexicographical order
  and save a lot of time.
  $\log_{26}N \ll N$!
\end{exm}

\begin{defn}
  \label{def:partialOrderAndPoset}
  A binary relation ``$\le$'' on some set ${\cal S}$
  is a \emph{partial order} on ${\cal S}$
  iff, $\forall a,b,c\in{\cal S}$,
  antisymmetry, transitivity, and reflexivity ($a\le a$)
  hold.\\
  A set equipped with a partial order
  is called a \emph{poset}.
\end{defn}

\begin{exm}
  The set of subsets of a set ${\cal S}$
  ordered by inclusion ``$\subseteq$.''
\end{exm}

\begin{exm}
  The natural numbers equipped with the relation of divisibility.
\end{exm}

\begin{exm}
  \label{exm:stuffToPutOn}
  The set of stuff you will put on your body every morning
  with the time ordered:
  undershorts, pants, belt, shirt, tie, jacket,
  socks, shoes, watch.
\end{exm}

\begin{rem}
  To learn something new essentially means
  to associate the new concept to certain firmly established concepts.
  These established fixture can be our daily experience.
  As a joke about Example \ref{exm:stuffToPutOn},
  common people put on undershorts first and then pants,
  but superman put on pants first and then undershorts.
\end{rem}

\begin{exm}
   \label{exm:inheritance}
  Inheritance (``is-a'' relation) is a partial order.
  $A \rightarrow B$ reads ``$B$ is a special type of $A$''.
\end{exm}

\begin{exm}
   \label{exm:composition}
  Composition (``has-a'' relation) is also a partial order.
  $A \leadsto B$ reads
  ``B \emph{has an} instance/object of A.''
\end{exm}

\begin{exm}
  Implication ``$\Rightarrow$''
  is a partial order on the set of logical statements.
\end{exm}

\begin{exm}
  The set of definitions, axioms,
  propositions, theorems, lemmas, etc., 
  is a poset with inheritance, composition, and implication.
  It is helpful to relate them with these partial orderings.
\end{exm}

\begin{defn}
  \label{def:UpperBoundAndMaximalElementOfPoset}
  An \emph{upper bound} of a subset $W$ of a poset $M$
  is an element $u\in M$ such that
  $x\le u$ for each $x\in W$.
  A \emph{maximal element} of a poset $M$
  is an $m\in M$ such that
  \begin{equation}
    \label{eq:maximalElementOfPoset}
    \forall x\in M,\ x\ge m \ \Rightarrow\ x=m.
  \end{equation}
\end{defn}

\begin{rem}
  Depending on $M$ and $W$,
  an upper bound of $W$ may or may not exist.
  Also,
  a poset may or may not have maximal elements.
\end{rem}

\begin{axm}[Zorn's lemma]
  \label{axm:ZornLemm}
  For a nonempty poset $M$,
  if every chain in $M$ has an upper bound,
  then $M$ has at least one maximal element.
\end{axm}

\begin{lem}[The Union Lemma]
  \label{lem:UnionLemma}
  Let $X$ be a set and $\mathcal{C}$ be a collection of subsets of $X$.
  Assume that for each $x\in X$,
  there is a set $A_x$ in $\mathcal{C}$
  such that $x\in A_x$. Then $\cup_{x\in X}A_x=X$. 
\end{lem}


\section{Functions}
\label{sec:functions}

\begin{defn}
  A \emph{function}/\emph{map}/\emph{mapping} $f$
  from ${\cal X}$ to ${\cal Y}$,
  written $f: {\cal X}\rightarrow {\cal Y}$ or ${\cal X}\mapsto {\cal Y}$,
  is a subset of the Cartesian product ${\cal X} \times {\cal Y}$
  satisfying that
  $\forall x\in {\cal X}$,
  there is exactly one $y\in {\cal Y}$
  s.t. $(x,y)\in {\cal X}\times {\cal Y}$.
  ${\cal X}$ and ${\cal Y}$ are
  the \emph{domain} and \emph{codomain} of $f$,
  respectively.
\end{defn}

\begin{rem}
The important thing in the above definition
 is the uniqueness of the pair $(x,y)$.
Why?
\end{rem}

\begin{defn}
  A \emph{binary function} or a \emph{binary operation} on a set ${\cal S}$
  is a map \mbox{${\cal S}\times{\cal S} \rightarrow {\cal S}$}.
\end{defn}

\begin{defn}
  A function $f:{\cal X}\rightarrow {\cal Y}$ is said to be
  \emph{injective} or \emph{one-to-one} iff
   \begin{equation}
     \forall x_1\in {\cal X},\, \forall x_2\in {\cal X},\ \ 
     x_1\ne x_2 \ \Rightarrow\ f(x_1)\ne f(x_2).
   \end{equation}
  It is \emph{surjective} or \emph{onto} iff
   \begin{equation}
     \forall y\in {\cal Y},\ \exists x\in {\cal X}, \text{ s.t. }
     y=f(x).
   \end{equation}
  It is \emph{bijective} iff it is both injective and surjective.
\end{defn}

\begin{defn}
  \label{def:preimage}
  The \emph{preimage of a set $U\subset Y$}
  (or the \emph{fiber over $U$ })
  under a function $f: X\rightarrow Y$ is 
   \begin{equation}
     \label{eq:preimage}
     f^{-1}(U) := \{ x\in X: f(x)\in U\}.
   \end{equation}
\end{defn}

\begin{lem}
  \label{lem:setOpPreservation}
  For $f: X\rightarrow Y$, the operation $f^{-1}$ preserves inclusions,
  unions, intersections, and differences of sets, 
  \begin{subequations}
    \label{eq:setOpFpre}
    \begin{align}
      B_0\subseteq B_1 &\Rightarrow f^{-1}(B_0)\subseteq f^{-1}(B_1),
      \\
      f^{-1}(B_0\cup B_1) &= f^{-1}(B_0) \cup f^{-1}(B_1),
      \\
      f^{-1}(B_0\cap B_1) &= f^{-1}(B_0) \cap f^{-1}(B_1),
      \\
      f^{-1}(B_0\setminus B_1) &= f^{-1}(B_0) \setminus f^{-1}(B_1), 
    \end{align}
  \end{subequations}
  while $f$ only preserves inclusions and unions, 
  \begin{subequations}
    \label{eq:setOpF}
    \begin{align}
      A_0\subseteq A_1 &\Rightarrow f(A_0)\subseteq f(A_1),
      \\
      f(A_0\cup A_1) &= f(A_0) \cup f(A_1),
      \\
      f(A_0\cap A_1) &\subseteq f(A_0) \cap f(A_1),
      \\
      f(A_0\setminus A_1) &\supseteq f(A_0) \setminus f(A_1),
    \end{align}
  \end{subequations}
  where the equalities in the last two equations hold
  if $f$ is injective.
\end{lem}
\begin{proof}
  We only prove the third equation in (\ref{eq:setOpF}).
  Suppose $y\in f(A_0\cap A_1)$.
  Then there exists $x\in A_0\cap A_1$ such that $f(x)=y$.
   $x\in A_0$ implies $y\in f(A_0)$
   while $x\in A_1$ implies $y\in f(A_1)$.
  Hence $y\in f(A_0)\cap f(A_1)$.
  The above argument can be reversed if $f$ is injective.
\end{proof}

\begin{exm}
  For $f(x)=x^2$, $A_0=[-1,0]$, $A_1=[0,1]$,
  we have $\{0\}=f(A_0\cap A_1)\subset f(A_0)\cap f(A_1)=A_1$.
\end{exm}

\begin{lem}
  \label{lem:preimageSubsetIdentites}
  For a map $f: X\rightarrow Y$, $A\subseteq X$,
  and $B\subseteq Y$,
  we have
  \begin{align}
    \label{eq:preimageSubsetIdentitesA}
    A &\subseteq f^{-1}(f(A)),
    \\
    \label{eq:preimageSubsetIdentitesffInvB}
    f(f^{-1}(B))&\subseteq B,
  \end{align}
  where the first inclusion
  is an equality if $f$ is injective 
  and the second is an equaility
  if $f$ is surjective or $B\subseteq f(X)$.
\end{lem}
\begin{proof}
  By (\ref{eq:preimage}), $a\in A$ implies
  $a\in f^{-1}(f(A))$.
  Conversely, $a\in f^{-1}(f(A))$ implies
   $f(a)\in f(A)$.
   $f$ being injective dictates $a\in A$.

  By (\ref{eq:preimage}), $b\in f(f^{-1}(B))$ implies
  $b\in B$.
  Furthermore, if $f$ is surjective or $B\subseteq f(X)$,
  then for any $b\in B$ we have $f^{-1}(\{b\})\ne \emptyset$
  and thus
  \begin{displaymath}
    b\in f(f^{-1}(\{b\})) \subseteq f(f^{-1}(B)).\qedhere
  \end{displaymath}
\end{proof}


%\section{Countable and uncountable sets}
\section{Cardinality of sets}
\label{sec:count-unco-sets}

\begin{defn}
  \label{def:countability}
  A set ${\cal S}$ is \emph{countably infinite}
   iff there exists a bijective function
   $f:{\cal S}\rightarrow \mathbb{N}^+$
   that maps ${\cal S}$ to $\mathbb{N}^+$.
  A set is \emph{countable}
  if it is either finite or countably infinite;
  it is \emph{uncountable}
  if it is not countable.
\end{defn}

\begin{exc}
  Are the integers countable?
  Are the rationals countable?
  Are the real numbers countable?
\end{exc}

\begin{defn}
  \label{def:CantorSet}
  The \emph{Cantor set} is a subset of $\mathbb{R}$
  given by $C:=\cap_{n=0}^{+\infty} C_n$
  where $C_0=[0,1]$
  and each $C_{n+1}$ is obtained
  by deleting from $C_n$ the open middle third
  of each closed interval. % of $F_n$
\end{defn}

\begin{rem}
  As a very intricate mathematical object, 
  the Cantor set is uncountable.
\end{rem}

\begin{thm}
  \label{thm:coutableSequenceUnion}
  Let $(E_n)_{n\in\mathbb{N}}$ be a sequence of countable sets.
  Then $S:=\cup_{n\in\mathbb{N}}E_n$ is countable.
\end{thm}

\begin{thm}
  \label{thm:uncountableSequenceOf0and1}
  Let $A$ be the set of all sequences whose elements
  are either 0 or 1.
  Then $A$ is uncountable. 
\end{thm}

%%% Local Variables:
%%% mode: latex
%%% TeX-master: "../numPDEs"
%%% End:
